\documentclass{article}
\usepackage[utf8]{inputenc}

\title{CSE3140 — Lab 0}
\author{Mike Medved, Shreya Seshadri}
\date{August 29th, 2022}

\usepackage{color}
\usepackage{amsthm}
\usepackage{amssymb} 
\usepackage{amsmath}
\usepackage[margin=1in]{geometry} 
\usepackage{listings}
\usepackage{xcolor}
\usepackage{minted}

\begin{document}

\maketitle

\section{Deliverables}

\subsection{Part 3: Directories, pwd, ls, man, cd, and mkdir}

The Q3 file for my group's VM was named \textbf{Q3.14323}. The file contained exactly 256 X's, and since each character is 1 byte long, the resultant file size was \textbf{256 bytes}.

\subsection{Part 4: Terminal Text Editors - Vim}

The password for the User with ID \#44444 in my group's VM file is \textbf{P63348U4020,P78930}. I accomplished finding this password by using the builtin search functionality in Vim. This was done by using the \textbf{/ (Slash)} operator and subsequently typing in the search target value, \textbf{U44444}. 

\subsection{Part 5: chmod + sudo}

In order to retrieve the contents of the Q5 file in my group's VM, I used the sudo command in tandem with the chmod command with the following parameters:

\begin{minted}{zsh}
sudo chmod +rw Q5
\end{minted}

$\hfill \break$
After executing the above command, I was granted read/write permissions to the file, which I was able to use in order to both (a) copy the file to the Solutions directory, and (b) get the contents to put them in this report:

$\hfill \break$
The contents of my group's file were: \textbf{V86213}

\subsection{Part 6: Grep, Redirection, and Pipes}

In order to find an arbitrary string according to a pattern, I used a regular expression in tandem with the grep command to filter out non-matching lines in the passwords file. The regular expression I used was \textbf{U6.*Y}, which matched against U6xxxxxY by using the "any" operator \textbf{. (Period)}. The command used is shown below:

\begin{minted}{bash}
cat Passwords.txt | grep U6.*Y
\end{minted}

$\hfill \break$
With this command, I found the resultant user ID: \textbf{U659183Y}.

\subsection{Part 7: Python}

In order to be able to run the Python file, Q7.py, I needed to give it read permissions, so I used chmod with sudo to accomplish this.

\begin{minted}{bash}
sudo chmod +r Q7.py
\end{minted}

$\hfill \break$
After granting it read permissions, I was able to execute the file using the python3 binary combined with a redirection operator in order to retrieve (and copy into the Solutions directory) the following output: \textbf{S877858783804296403038}

\end{document}
